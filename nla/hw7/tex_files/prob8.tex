\documentclass{article}
\usepackage{amsmath,nccmath}
\usepackage[landscape]{geometry}

\begin{document}
\section*{Problem 8}
% PART i
i) To find the asset allocation of the tangency portfolio, we use the minimum variance portfolio from tangency portfolio function (p256 from the book) with:
\begin{equation*}
\Sigma_{R}=% latex table generated in R 3.1.1 by xtable 1.7-4 package
% Mon Mar 23 19:34:59 2015
\begin{bmatrix}{}
 0.090000 & 0.010000 & 0.030000 & -0.015000 \\ 
  0.010000 & 0.062500 & -0.020000 & -0.010000 \\ 
  0.030000 & -0.020000 & 0.122500 & 0.020000 \\ 
  -0.015000 & -0.010000 & 0.020000 & 0.057600 \\ 
  \end{bmatrix},
\mu = \begin{bmatrix}{}
 0.040000 \\ 
  0.035000 \\ 
  0.050000 \\ 
  0.034000 \\ 
  \end{bmatrix}
\end{equation*}
We obtain that:
\begin{equation*}
w_{T} = % latex table generated in R 3.1.1 by xtable 1.7-4 package
% Mon Mar 23 19:37:10 2015
\begin{bmatrix}{}
 0.173699 \\ 
  0.338145 \\ 
  0.173604 \\ 
  0.314552 \\ 
  \end{bmatrix}, 
\mu_{portfolio} = 0.038158,
\sigma_{portfolio} = 0.135302,
{Sharpe Ratio}_{portfolio} = 0.208112
\end{equation*}
\vspace{5mm} \\
% PART ii
%%
ii) To find the asset allocation for a minimum variance portfolio with 3\% expected return, we use Table 9.1 on p255 with:
\begin{equation*}
\Sigma_{R}=% latex table generated in R 3.1.1 by xtable 1.7-4 package
% Mon Mar 23 19:34:59 2015
\begin{bmatrix}{}
 0.090000 & 0.010000 & 0.030000 & -0.015000 \\ 
  0.010000 & 0.062500 & -0.020000 & -0.010000 \\ 
  0.030000 & -0.020000 & 0.122500 & 0.020000 \\ 
  -0.015000 & -0.010000 & 0.020000 & 0.057600 \\ 
  \end{bmatrix},
\mu = \begin{bmatrix}{}
 0.040000 \\ 
  0.035000 \\ 
  0.050000 \\ 
  0.034000 \\ 
  \end{bmatrix}
\end{equation*}
We obtain that:
\begin{equation*}
w_{min} =% latex table generated in R 3.1.1 by xtable 1.7-4 package
% Mon Mar 23 19:43:40 2015
\begin{bmatrix}{}
 0.123374 \\ 
  0.240177 \\ 
  0.123307 \\ 
  0.223419 \\ 
  \end{bmatrix},
w_{min,cash} = 0.289722,
\sigma_{portfolio} = 0.096102,
{Sharpe Ratio}_{portfolio} = 0.208112
\end{equation*}
\vspace{5mm} \\
% PART iii
%%
iii) To find the asset allocation for a maximum return portfolio with 27\% standard deviation, we use Table 9.3 on p257 with:
\begin{equation*}
\Sigma_{R}=% latex table generated in R 3.1.1 by xtable 1.7-4 package
% Mon Mar 23 19:34:59 2015
\begin{bmatrix}{}
 0.090000 & 0.010000 & 0.030000 & -0.015000 \\ 
  0.010000 & 0.062500 & -0.020000 & -0.010000 \\ 
  0.030000 & -0.020000 & 0.122500 & 0.020000 \\ 
  -0.015000 & -0.010000 & 0.020000 & 0.057600 \\ 
  \end{bmatrix},
\mu = \begin{bmatrix}{}
 0.040000 \\ 
  0.035000 \\ 
  0.050000 \\ 
  0.034000 \\ 
  \end{bmatrix}
\end{equation*}
We obtain that:
\begin{equation*}
w_{max} =% latex table generated in R 3.1.1 by xtable 1.7-4 package
% Mon Mar 23 19:47:44 2015
\begin{bmatrix}{}
 0.346622 \\ 
  0.674782 \\ 
  0.346433 \\ 
  0.627700 \\ 
  \end{bmatrix},
w_{max,cash} = -0.995537,
\sigma_{portfolio} = 0.066190,
{Sharpe Ratio}_{portfolio} = 0.208112
\end{equation*}
\vspace{5mm} \\
% PART iv
%%
iv) To find the asset allocation for a minimum variance portfolio fully invested in both assets, we use 9.76 and 9.77 on p272 with:
\begin{equation*}
\Sigma_{R}=% latex table generated in R 3.1.1 by xtable 1.7-4 package
% Mon Mar 23 19:34:59 2015
\begin{bmatrix}{}
 0.090000 & 0.010000 & 0.030000 & -0.015000 \\ 
  0.010000 & 0.062500 & -0.020000 & -0.010000 \\ 
  0.030000 & -0.020000 & 0.122500 & 0.020000 \\ 
  -0.015000 & -0.010000 & 0.020000 & 0.057600 \\ 
  \end{bmatrix}
\end{equation*}
We obtain that:
\begin{equation*}
w_{no\_cash} =% latex table generated in R 3.1.1 by xtable 1.7-4 package
% Mon Mar 23 19:52:30 2015
\begin{bmatrix}{}
 0.190503 \\ 
  0.339011 \\ 
  0.089277 \\ 
  0.381209 \\ 
  \end{bmatrix},
{Sharpe Ratio}_{portfolio} = 0.203450
\end{equation*}
\end{document}
